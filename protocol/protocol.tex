\documentclass{article}
\usepackage[hidelinks]{hyperref}
\usepackage{bookmark}
\usepackage{graphicx}
\usepackage{amsmath}
\usepackage{amssymb}
\usepackage{booktabs}
\usepackage[margin=1in]{geometry}
\usepackage{fancyhdr}
\usepackage{parskip}
\usepackage[round]{natbib}
\bibliographystyle{apalike}

\title{Association of Stroke and Traumatic Brain Injury with Healthcare Access and Utilization in the U.S. Adult Population: A Propensity Score--Matched Analysis of NHANES 2011--2014}
\author{Mike Baiocchi, Jonathan Pipping, Andrea Schneider, 
Ashil Srivastava and Dylan Small}
\date{}

\begin{document}

% Set up running headers and footers∂
\pagestyle{fancy}
\fancyhf{}
\fancyhead[L]{\leftmark}
\fancyhead[R]{Protocol}
\fancyfoot[C]{\thepage}
\renewcommand{\headrulewidth}{0.4pt}
\renewcommand{\footrulewidth}{0pt}

\maketitle

\tableofcontents

\pagebreak

\section{Title and Abstract}

\subsection{Title}

Association of Stroke and Traumatic Brain Injury with Healthcare Access and Utilization in the U.S. Adult Population: A Propensity Score--Matched Analysis of NHANES 2011--2014

\subsection{Abstract}

\textbf{Background:} Traumatic brain injury (TBI) and stroke are common neurologic conditions associated with substantial morbidity and mortality in the United States. Both conditions can have long-term cognitive, functional, and psychosocial consequences that may increase the need for ongoing healthcare. At the same time, barriers to healthcare access and utilization may worsen outcomes among individuals with a history of TBI or stroke.

\textbf{Objectives:} This study will estimate the association between self-reported lifetime history of (1) stroke and (2) traumatic brain injury (TBI) with loss of consciousness and multiple indicators of healthcare access and utilization in a nationally representative sample of U.S. adults.

\textbf{Data and Setting:} We will use data from the 2011--2012 and 2013--2014 cycles of the National Health and Nutrition Examination Survey (NHANES), a complex, multistage probability sample of the noninstitutionalized U.S. civilian population.

\textbf{Design:} This is an observational, cross-sectional study using existing survey data. For each exposure (stroke and TBI), we construct separate analytic cohorts and estimate propensity scores for exposure as a function of demographic, socioeconomic, health behavior, cardiovascular risk, and survey design variables (strata, primary sampling unit, and interview weights), along with missingness indicators. We perform optimal $1\!:\!4$ propensity score matching (selected after evaluating ratios 1:1--1:6) with a caliper of $0.2$ pooled standard deviations on the logit of the propensity score and assess covariate balance using standardized mean differences and love plots (Section~6).

\textbf{Primary Outcomes:} The primary outcomes are (1) having a usual place to go for healthcare and (2) having any health insurance coverage. Secondary outcomes include type of insurance (public vs.\ private), type of usual source of care, time since last healthcare visit, mental health service use, hospitalizations, and prescription coverage.

\textbf{Analytic Plan:} Within the matched samples, we will estimate the population average treatment effect on the treated (PATT), i.e., the effect among adults with a history of stroke or TBI, using survey-weighted models that incorporate both NHANES sampling weights and the matched design, following the framework of \citet{dugoff2014} for combining propensity score methods with complex survey data.

\textbf{Results and Conclusions:} This document describes the protocol and planned analyses. Numerical results and final conclusions will be added once analyses are completed.

%========================
\section{Introduction}
%========================

Traumatic brain injury (TBI) and stroke are major causes of death and disability in the United States. TBI, typically resulting from a bump, blow, or jolt to the head---for example from falls or motor vehicle crashes---ranges in severity from mild concussions to severe injuries causing permanent neurologic impairment. Stroke, caused by interruption of blood flow to the brain (ischemic) or rupture of a cerebral blood vessel (hemorrhagic), similarly leads to significant cognitive and functional impairment and often requires intensive acute and chronic care.

In the United States, hundreds of thousands of adults each year experience stroke, and many die from TBI-related injuries, with many more living with long-term aftereffects. Adequate access to healthcare, including continuous insurance coverage, a usual source of care, and timely primary and specialty services, is likely critical to optimizing outcomes among people with a history of stroke or TBI. Conversely, gaps in insurance coverage, lack of a usual source of care, and delayed or forgone care may exacerbate disability and increase the risk of preventable complications.

Despite the plausibility of a strong link between severe neurologic injury and subsequent healthcare access and utilization, evidence on this association at the population level using nationally representative data with comprehensive covariate adjustment remains limited. While some studies have examined healthcare utilization following stroke or TBI, existing work often relies on single health systems, selected subpopulations (e.g., hospitalized patients), or administrative data that may not simultaneously capture neurologic history, detailed healthcare access measures, and rich demographic and socioeconomic covariates needed for robust causal inference.

The National Health and Nutrition Examination Survey (NHANES) provides an opportunity to address this gap. NHANES is a nationally representative, multistage probability survey of the noninstitutionalized U.S. civilian population that collects standardized information on demographics, health behaviors, chronic conditions, health insurance, and healthcare utilization. In the 2011--2012 and 2013--2014 survey cycles, NHANES asked respondents about lifetime history of stroke and, in adults aged 40 years and older, history of head injury resulting in loss of consciousness.

Using these data, this study will:

\begin{enumerate}
    \item Quantify how healthcare access and utilization differ between adults with and without a history of stroke and between adults with and without a history of TBI.

    \item Evaluate whether these associations vary across demographic and socioeconomic subgroups (e.g., age, sex, race/ethnicity, education, income).

    \item Implement propensity score methods combined with NHANES survey weights, following \citet{dugoff2014}, to obtain effect estimates that are both adjusted for observed confounding and generalizable to the national target population.

\end{enumerate}

By doing so, the study aims to inform whether individuals with prior stroke or TBI experience systematic differences in access to and use of healthcare services relative to the broader U.S. adult population.

%========================
\section{Research Question}
%========================

\subsection{Primary Research Question}

Among noninstitutionalized U.S. adults represented by NHANES 2011--2014, what is the association between self-reported lifetime history of:

\begin{enumerate}
    \item Stroke, and

    \item Traumatic brain injury (TBI) with loss of consciousness

\end{enumerate}

and multiple indicators of healthcare access and utilization, including having a usual source of care and health insurance coverage?

Formally, for each exposure (stroke, TBI) we seek to estimate the \emph{population average treatment effect on the treated} (PATT)---the difference in healthcare access and utilization outcomes between:

\begin{itemize}
    \item adults who report a history of stroke (or TBI), and

    \item the same population, had they not experienced stroke (or TBI),

\end{itemize}

generalized to the NHANES target population via survey weighting.

\subsection{Hypotheses}

\textbf{Primary hypotheses}

\begin{itemize}
    \item Adults with a history of TBI are less likely to have adequate healthcare access and utilization (e.g., less likely to have a usual place for care, more likely to lack coverage or have longer time since last visit).

    \item Adults with a history of stroke are more likely to have healthcare access and utilize healthcare services (e.g., more likely to have a usual source of care and insurance) compared with adults without stroke, reflecting increased contact with the healthcare system after a major vascular event.

\end{itemize}

\textbf{Secondary hypothesis}

\begin{itemize}
    \item The magnitude and direction of these associations differ across subgroups defined by age, sex, race/ethnicity, education, and income-to-poverty ratio (i.e., effect modification by demographic and socioeconomic characteristics).

\end{itemize}

\subsection{Objectives}

\begin{enumerate}
    \item Describe healthcare access and utilization indicators among U.S. adults overall and stratified by history of stroke and TBI.

    \item Estimate the association (PATT) between stroke history and each healthcare access/utilization outcome, adjusting for observed confounders using propensity score matching combined with survey design-based analysis.

    \item Estimate the analogous association for TBI.

    \item Explore effect modification by key demographic and socioeconomic variables.

\end{enumerate}

\subsection{Generalization Target}

The target population for inference is the U.S.\ noninstitutionalized civilian population represented by NHANES 2011--2014. Findings are intended to apply to:

\begin{itemize}
    \item Adults aged $\geq 20$ years for the stroke analysis (the age range for which stroke history was collected), and

    \item Adults aged $\geq 40$ years for the TBI analysis (the age range for which the TBI module was administered).

\end{itemize}

Findings will not generalize to:

\begin{itemize}
    \item Institutionalized populations (e.g., nursing homes, long-term care facilities),

    \item Active-duty military personnel, or

    \item People experiencing homelessness or others outside the NHANES sampling frame.

\end{itemize}

%========================
\section{Inclusion and Exclusion Criteria}
%========================

\subsection{Data Source}

We will construct analytic cohorts from the combined 2011--2012 and 2013--2014 NHANES cycles. These cycles include:

\begin{itemize}
    \item Demographics (DEMO),

    \item Health insurance and access (HIQ, HUQ),

    \item Medical conditions including stroke (MCQ),

    \item Traumatic brain injury (CSQ; administered to adults $\geq 40$ years),

    \item Alcohol use (ALQ), smoking (SMQ), hypertension (BPQ), and diabetes (DIQ).

\end{itemize}

These components will be merged by respondent identifier (SEQN) and survey cycle.

\subsection{Inclusion Criteria}

We will define separate analytic samples for the stroke and TBI analyses.

\paragraph{Stroke analysis}

\begin{itemize}
    \item Adults aged $\geq 20$ years at the time of the NHANES examination.

    \item Nonmissing response to the stroke history question (MCQ160F: ``Has a doctor or other health professional ever told you that you had a stroke?'' coded as yes/no).

    \item Nonmissing data on the primary outcomes (usual source of care and health insurance coverage).

    \item Availability of NHANES design variables (strata, primary sampling unit) and interview sampling weights (WTINT2YR).

\end{itemize}

\paragraph{TBI analysis}

\begin{itemize}
    \item Adults aged $\geq 40$ years (NHANES eligibility for the TBI module).

    \item Nonmissing response to the TBI question (CSQ240: ``Have you ever had loss of consciousness because of a head injury?'' yes/no).

    \item Nonmissing data on the primary outcomes.

    \item Availability of design variables and interview sampling weights.

\end{itemize}

\subsection{Exclusion Criteria}

We will exclude:

\begin{itemize}
    \item Respondents with missing or indeterminate exposure status (stroke history or TBI),

    \item Respondents missing all primary outcomes,

    \item Respondents without valid sampling weights or design variables required for survey analysis,

    \item Participants outside the age ranges for each analytic cohort (e.g., $<20$ for stroke analysis, $<40$ for TBI analysis).

\end{itemize}

We will not exclude participants solely based on missing covariate values used for propensity score estimation; instead, we will address covariate missingness as described in Section~\ref{sec:design} (missingness indicators and deterministic value assignment).

\subsection{Data Counts and Flow}

In the final manuscript, we will report:

\begin{itemize}
    \item The total number of NHANES participants in 2011--2014,

    \item Counts excluded at each step (e.g., ineligible age, missing exposure, missing outcomes, missing design variables),

    \item Number of exposed and unexposed participants entering the propensity score models, and

    \item Number retained in the matched samples for each exposure.

\end{itemize}

These will be summarized in a CONSORT-style flow diagram, as recommended in observational protocol guidance.

%========================
\section{Study Outcomes}
%========================

\subsection{Exposures}

We consider two separate binary exposures, each analyzed in its own matched cohort.

\paragraph{History of stroke}

\begin{itemize}
    \item Derived from NHANES item MCQ160F (``Ever told you had a stroke?'').

    \item Coded as: $1 = \text{Yes}$ (stroke\_exposed), $0 = \text{No}$.

\end{itemize}

\paragraph{History of traumatic brain injury (TBI)}

\begin{itemize}
    \item Derived from NHANES item CSQ240 (``Ever had a loss of consciousness because of a head injury?'').

    \item Coded as: $1 = \text{Yes}$ (tbi\_exposed), $0 = \text{No}$.

\end{itemize}

\subsection{Primary Outcomes}

We will define two primary healthcare access outcomes (binary).

\paragraph{Usual place to go for healthcare}

Derived from NHANES item HUQ030 (``Is there a place you usually go when you are sick or need advice about your health?''). Coded as: $1 = \text{Has a usual place}$ (response = 1 or 3), $0 = \text{No usual place}$ (response = 2).

\paragraph{Any health insurance coverage}

Derived from NHANES item HIQ011 (``Are you covered by health insurance or some other kind of health care plan?''). Coded as: $1 = \text{Any coverage}$ (public or private), $0 = \text{Uninsured at the time of interview}$ (response = 2).

These primary outcomes directly reflect core dimensions of healthcare access: having a regular source of care and being insured.

\subsection{Secondary Outcomes}

Secondary outcomes will characterize additional aspects of access and utilization:

\begin{itemize}
    \item Type of insurance plan (e.g., government vs.\ private coverage among those insured),

    \item Type of usual source of care (e.g., doctor's office/clinic, hospital outpatient clinic, emergency department, other),

    \item Time since last healthcare visit (categorical measure based on HUQ visit timing, e.g., $\leq 6$ months, 6--12 months, $>12$ months),

    \item Mental health service use (visit to a mental health professional in the past year; yes/no),

    \item Hospital utilization (number of overnight hospital stays in the past year, modeled as count or categorized),

    \item Insurance continuity and prescription coverage (any gaps in health insurance in the past year; whether the plan covers prescription medications).

\end{itemize}

All secondary outcomes will be coded in a way that aligns with NHANES public documentation and allows for interpretable contrasts between exposed and unexposed groups.

\subsection{Marginal Distributions and Descriptive Summaries}

Before matching, we will:

\begin{itemize}
    \item Present weighted marginal distributions (e.g., proportions, means) of all primary and secondary outcomes in the overall sample and stratified by exposure status (stroke/TBI yes/no),

    \item Produce tables and figures (e.g., bar charts, histograms) to describe the distribution of outcomes across survey cycles (2011--2012 vs.\ 2013--2014), consistent with the shell tables.

\end{itemize}

These summaries will provide context for the matched analyses and help assess whether the matched sample remains representative of the target population.

%========================
\section{Study Design}
\label{sec:design}
%========================

\subsection{Overview and Primary Design Features}

This is an observational, cross-sectional study using existing NHANES data to estimate the association between lifetime neurologic injury (stroke, TBI) and healthcare access/utilization. The principal design elements are:

\begin{enumerate}
    \item Use of a nationally representative complex survey (NHANES) to enable population-level inference.

    \item Propensity score matching to reduce confounding by observed covariates.

    \item Combination of propensity score methods with survey weighting and design-based analysis, following \citet{dugoff2014}, to obtain effect estimates that are both confounding-adjusted and generalizable to the NHANES target population (PATT).

\end{enumerate}

We will conduct two parallel analyses: one with stroke as the exposure, one with TBI as the exposure.

\subsection{Covariates: Observed Confounders and Effect Modifiers}

Based on prior literature and data availability, we identify the following as key observed confounders and potential effect modifiers:

\begin{itemize}
    \item \textbf{Demographic characteristics:} age (RIDAGEYR), sex (RIAGENDR), race/ethnicity (RIDRETH3).

    \item \textbf{Socioeconomic status:} education level (DMDEDUC2), income-to-poverty ratio (INDFMPIR).

    \item \textbf{Health behaviors and risk factors:} alcohol use and potential alcohol abuse (derived from ALQ151, ALQ141Q), smoking status (derived from SMQ020 and SMQ040; categories: every day, some days, not at all), hypertension (BPQ020), diabetes (DIQ010, including borderline as ``yes'').

    \item \textbf{History of stroke:} for the TBI analysis, we will include stroke\_history (MCQ160F) as a covariate, given its strong relation to both TBI and healthcare utilization.

    \item \textbf{Survey design characteristics:} primary sampling unit (SDMVPSU), stratum (SDMVSTRA), and NHANES interview sampling weight (WTINT2YR) will be included as predictors in the propensity score model, consistent with \citet{dugoff2014}'s recommendations for Stage 1.

\end{itemize}

We will treat age, sex, race/ethnicity, education, and income as candidate effect modifiers and will explore interactions in secondary analyses.

\subsection{Potential Unobserved Confounders}

We acknowledge that some important determinants of both exposure and outcomes may be unmeasured or imperfectly measured in NHANES, including:

\begin{itemize}
    \item Severity, timing, and mechanism of stroke or TBI,

    \item Cognitive impairment severity, functional status, and disability,

    \item Social support and detailed health system characteristics.

\end{itemize}

These factors may induce residual confounding. We plan to address robustness to unmeasured confounding via sensitivity analyses (see planned Section~8), for example gamma-based sensitivity analyses for matched observational studies.

\subsection{Handling Missing and Corrupted Data}

We will use a structured approach to missing data:

\begin{enumerate}
    \item \textbf{Missingness indicators.} For each covariate with missing values (e.g., RIDAGEYR, RIAGENDR, RIDRETH3, DMDEDUC2, INDFMPIR, alcohol\_abuse, smoking\_status, hypertension, diabetes, stroke\_history), we will create a binary indicator (e.g., \texttt{RIDAGEYR\_missing}) equal to 1 if the original value is missing and 0 otherwise. These indicators will be included in the propensity score model so that matching explicitly accounts for patterns of missingness.

    \item \textbf{Deterministic value assignment for missing entries.} To ensure that all records have well-defined covariate values for propensity score estimation and matching, we will assign a fixed, internally consistent placeholder value to missing entries, while retaining the corresponding missingness indicators described above. Specifically, continuous variables (e.g., age, income-to-poverty ratio) will have missing entries replaced with the analytic-sample mean; categorical variables (e.g., sex, race/ethnicity, education, smoking\_status) will include an explicit ``Missing'' category; and binary health conditions (alcohol\_abuse, hypertension, diabetes, stroke\_history) will have missing entries replaced with the modal category. This approach is intended to preserve the analytic sample and allow the propensity score to adjust for missingness through the indicator variables, rather than relying on a model-based missing-data procedure.

    \item \textbf{Exposures and primary outcomes.} Exposure variables (stroke\_exposed, tbi\_exposed) and primary outcomes will not be filled in; records with missing exposure or missing primary outcomes will be excluded from the analytic sample as described in Section~4.

\end{enumerate}

Missingness indicators will then be included as covariates in the propensity score models, allowing the propensity score to account for systematic patterns of missingness in covariates.

\subsection{Propensity Score Estimation}

For each exposure (stroke and TBI), we will estimate propensity scores using logistic regression. The propensity score models will include all measured covariates (demographic characteristics, socioeconomic status, health behaviors and risk factors, and missingness indicators), along with the NHANES survey weights (WTINT2YR) and design variables (SDMVPSU, SDMVSTRA). For the TBI analysis, we will also include stroke history as a covariate. This approach is consistent with \citet{dugoff2014}'s Stage-1 recommendation for incorporating survey design information in propensity scores. We will predict propensity scores and their logits for all individuals with nonmissing exposure status.

\subsection{Matching Algorithm}

We will implement optimal propensity score matching using the \texttt{MatchIt} package in R. For each exposure, we will:

\begin{enumerate}
    \item Restrict to participants with nonmissing exposure status (stroke data, TBI data).

    \item Fit a logistic regression model to estimate propensity scores for all participants with nonmissing exposure status.

    \item Compute the pooled standard deviation of the logit propensity score among treated and control participants.

    \item Define a caliper as $0.2$ times the pooled standard deviation of the logit propensity score.

    \item Construct a distance matrix based on absolute differences in logit propensity score between treated and control units.

    \item Apply the caliper by adding a large penalty (1000) to distances exceeding the caliper.

    \item Perform optimal $1\!:\!k$ matching ($k = 1,\dots,6$) without replacement using \texttt{MatchIt}'s \texttt{method = "optimal"} and the pre-computed distance matrix.

\end{enumerate}

For each matching ratio (1:1 through 1:6), we will:

\begin{itemize}
    \item Extract the matched dataset,

    \item Record the number of matched treated and control individuals and any treated units discarded due to the caliper, and

    \item Save matched datasets for later analysis.

\end{itemize}

This design primarily targets the average treatment effect on the treated (ATT) in the matched sample; in combination with survey weights in the outcome analysis, we will interpret estimates as approximating the population average treatment effect on the treated (PATT) in the NHANES target population.

\subsection{Covariate Balance and Diagnostics}

We will assess covariate balance using the \texttt{cobalt} package in R. For each matched dataset, we will:

\begin{itemize}
    \item Compute standardized mean differences (SMD) before and after matching using \texttt{bal.tab},

    \item Generate love plots summarizing covariate balance pre- and post-matching,

    \item Summarize residual imbalance by counting covariates with $|\text{SMD}| > 0.05$, $0.10$, and $0.20$, and report the maximum $|\text{SMD}|$ and the percentage of covariates with $|\text{SMD}| > 0.10$.

\end{itemize}

We will select a final matching ratio (e.g., 1:1, 1:2, etc.) for each exposure based on:

\begin{itemize}
    \item Achieving SMD $< 0.10$ for all or nearly all covariates,

    \item Minimizing the number of treated units discarded, and

    \item Preserving adequate sample size for stable estimation.

\end{itemize}

\paragraph{Chosen matching design.}
We compared candidate fixed-ratio matches from 1:1 through 1:6 and selected the final ratio using only design-stage diagnostics (i.e., covariate balance, overlap, and retention), without examining any outcome variables. Based on the criteria above, we selected \textbf{1:4 matching} for both exposures because it achieved the best overall trade-off between covariate balance and retention of treated participants.

For stroke, the 1:4 matched sample met the balance criterion across all covariates (max $|\text{SMD}| = 0.043$; 0 covariates with $|\text{SMD}| > 0.10$), retained all 431 treated participants, and yielded a total matched sample size of 2,155 for outcome modeling. For TBI, the 1:4 matched sample also met the balance criterion (max $|\text{SMD}| = 0.095$; 0 covariates with $|\text{SMD}| > 0.10$), retained all 948 treated participants, and yielded a total matched sample size of 4,740. All outcome analyses (Section~7) will be conducted using these 1:4 matched datasets.

\paragraph{Love plots (1:4 matching).}
Figure~\ref{fig:love-stroke} and Figure~\ref{fig:love-tbi} display standardized mean differences (SMD) for each covariate before and after 1:4 propensity score matching for the stroke and TBI analyses, respectively. Values closer to zero indicate improved balance after matching. The vertical reference line at 0.10 denotes the a priori balance threshold. Love plots (and balance tables) for all evaluated ratios (1:1--1:6) will be summarized in the final manuscript and/or supplement.

\begin{figure}[htbp]
\centering
\includegraphics[width=0.8\textwidth]{../matching/love_stroke_1_4.png}
\caption{Covariate balance before and after 1:4 propensity score matching: stroke vs.\ no stroke. The vertical line at 0.10 indicates the balance threshold.}
\label{fig:love-stroke}
\end{figure}

\begin{figure}[htbp]
\centering
\includegraphics[width=0.8\textwidth]{../matching/love_tbi_1_4.png}
\caption{Covariate balance before and after 1:4 propensity score matching: TBI vs.\ no TBI. The vertical line at 0.10 indicates the balance threshold.}
\label{fig:love-tbi}
\end{figure}

Diagnostic plots and summary statistics will be included in the final manuscript to demonstrate successful balance in the matched samples.

\subsection{Negative control balance check}
\label{sec:neg-control}

As an additional post-matching diagnostic for residual confounding, we will evaluate balance on a prespecified \emph{negative control} variable that is not used in the propensity score (PS) model or the matching procedure: lifetime marijuana ever-use.

\paragraph{Negative control definition.}
We define a binary indicator $Z_i$ for whether respondent $i$ reports having ever tried marijuana/hashish, derived from the NHANES Drug Use Questionnaire item \texttt{DUQ200} (``Ever used marijuana or hashish?'').\footnote{NHANES DUQ is administered only to a restricted age range (e.g., through age 69 in 2011--2014), so this diagnostic will be assessed within the subset for whom \texttt{DUQ200} is collected and nonmissing.} We code $Z_i=1$ for ``Yes'' and $Z_i=0$ for ``No''; responses such as ``Refused,'' ``Don't know,'' or not administered will be treated as missing for this diagnostic.

\paragraph{Rationale.}
Marijuana ever-use is plausibly associated with unmeasured behavioral and social factors (e.g., risk-taking, policing/incarceration exposure, social instability) that may also relate to healthcare access and utilization, but it is not a study exposure, not a study outcome, and is not included in the PS model. Therefore, imbalance on $Z_i$ after matching would serve as a falsification-style indicator of residual confounding or incomplete balance on correlated, unmeasured constructs.

\paragraph{Diagnostic procedure.}
Within each final matched dataset (stroke; TBI), and restricting to respondents with observed $Z_i$, we will:
\begin{enumerate}
    \item Compute the standardized mean difference (SMD) for $Z_i$ between exposed and matched unexposed participants, using the same analysis weights as the primary matched analyses (i.e., the product of the rescaled NHANES weight and the matching weight).
    \item Fit a survey-weighted logistic regression in the matched sample with $Z_i$ as the dependent variable and exposure indicator $A_i$ as the sole predictor:
    \[
    \text{logit}\{\Pr(Z_i = 1)\} = \alpha_0 + \alpha_1 A_i,
    \]
    and report $\exp(\alpha_1)$ with a 95\% confidence interval.
\end{enumerate}
We will interpret successful negative-control balance as SMD $<0.10$ and an odds ratio close to 1 (i.e., no meaningful association between exposure and marijuana ever-use in the matched sample).

\paragraph{Prespecified response to imbalance.}
If the negative control diagnostic indicates meaningful imbalance (e.g., SMD $\ge 0.10$ or an odds ratio materially different from 1 with a narrow confidence interval), we will treat this as evidence that the current PS specification/matching parameters may not have adequately balanced constructs correlated with unmeasured confounding. Without examining any primary or secondary outcome variables, we will consider design-stage revisions such as: (i) modifying the PS model to include additional pre-exposure behavioral covariates available in NHANES, (ii) tightening the caliper, and/or (iii) adjusting the matching ratio, and then re-assessing covariate balance. The final negative control diagnostic results will be reported alongside the main balance diagnostics.

\subsection{Outcome Analysis and Bias Control}

Following \citet{dugoff2014}, the outcome analysis will:

\begin{itemize}
    \item Use survey-weighted regression models (e.g., logistic regression for binary outcomes, Poisson/negative binomial for counts) applied to the matched data,

    \item Incorporate NHANES sampling weights and design variables (strata and PSU) to ensure proper variance estimation and population-level inference, and

    \item Reflect the matching design (by using the matched sample and appropriate weights).

\end{itemize}

For each exposure and outcome, the primary estimand will be the PATT---the difference in mean outcomes between exposed and unexposed, among those exposed in the target population.

This design addresses bias from observed covariates by:

\begin{itemize}
    \item Including a rich set of confounders and missingness indicators in the propensity score model,

    \item Ensuring good covariate balance after matching, and

    \item Using survey design-based inference to retain generalizability.

\end{itemize}

Unobserved confounding will be addressed in planned sensitivity analyses (Section~8).

\subsection{Implementation}

All data management and analyses will be implemented in R, using the following core packages: \texttt{haven} and \texttt{tidyverse} for data ingestion and merging; \texttt{readr} and \texttt{tableone} for descriptive summaries; \texttt{MatchIt} for propensity score matching; and \texttt{cobalt} for covariate balance diagnostics.


%========================
\section{Study Analysis Plan}
%========================

\subsection{Primary Analysis}

Within each matched sample (stroke vs.\ no stroke; TBI vs.\ no TBI), we estimate the association between exposure status and each healthcare access/utilization outcome using interview-weighted regression models.

\paragraph{Model specification}

For the two primary outcomes, we fit survey-weighted logistic regression models of the form:
\[
\text{logit}\{\Pr(Y_i = 1)\} = \beta_0 + \beta_1 A_i,
\]
where $A_i$ is an indicator for exposure (stroke or TBI). Because matching is intended to balance observed covariates, the primary model will include only the exposure indicator. If any prespecified covariate remains meaningfully imbalanced after matching (absolute standardized mean difference $>0.10$), we will include that covariate in an adjusted outcome model as a robustness check.

\paragraph{Secondary outcomes}

For secondary outcomes, we will use survey-weighted generalized linear models within the matched samples, choosing the link function and distribution to align with the outcome type. The primary predictor remains the exposure indicator $A_i$ (stroke or TBI), and the primary secondary-outcome models will include only $A_i$; as with the primary outcomes, if any covariate remains meaningfully imbalanced after matching (absolute SMD $>0.10$), we will fit an adjusted model including those covariates as a robustness check.

\textit{Binary secondary outcomes.} For binary secondary outcomes (e.g., public vs.\ private insurance among those insured; any mental health visit in the past year; any gap in insurance in the past year; prescription coverage; mental-health service coverage), we will fit survey-weighted logistic regression models:
\[
\text{logit}\{\Pr(Y_i = 1)\} = \gamma_0 + \gamma_1 A_i,
\]
and report $\exp(\gamma_1)$ as an odds ratio with a 95\% confidence interval, along with marginal predicted probabilities under $A=1$ and $A=0$ and their difference.

\textit{Categorical secondary outcomes.} For categorical secondary outcomes with more than two levels (e.g., type of usual source of care; time since last healthcare visit), we will fit survey-weighted multinomial logistic regression models. For multinomial models, we will report relative risk ratios (or odds ratios, depending on parameterization) comparing each non-reference category to the reference, with 95\% confidence intervals, and will present marginal predicted category probabilities for interpretability.

\textit{Count secondary outcomes.} For count outcomes (e.g., number of overnight hospital stays in the past year), we will fit survey-weighted count regression models. The default will be a Poisson model with log link,
\[
\log\{\mathbb{E}(Y_i)\} = \delta_0 + \delta_1 A_i,
\]
and we will assess overdispersion; if overdispersion is present, we will use a negative binomial model or (if necessary) a categorized version of the count (e.g., 0 vs.\ $\ge 1$, or 0 / 1 / $\ge 2$) analyzed via logistic or multinomial methods. We will report rate ratios (or odds ratios for categorized versions) and 95\% confidence intervals, and will provide marginal predicted means (or probabilities) under $A=1$ and $A=0$.

All secondary outcome analyses are considered supportive and hypothesis-generating relative to the primary outcomes; we will nonetheless apply multiplicity control to the secondary-outcome family as described below.

\paragraph{Weights and generalization}

Analyses will incorporate NHANES interview sampling weights and complex survey design variables (strata and primary sampling unit). Following \citet{dugoff2014}, we will fit design-based, weighted regression models within the matched sample to obtain estimates that generalize to the NHANES target population. For $1!:!k$ matched samples, we will use the matching weights output by the matching procedure. Because we pool two NHANES 2-year cycles (2011--2012 and 2013--2014), we rescale the 2-year interview weight by the number of cycles; specifically, we use $w_i^{NH}=\texttt{WTINT2YR}/2$. The final analysis weight is then
\[w_i = w_i^{NH}\cdot w_i^M\]
where $w_i^{M}$ denotes the matching weight for individual $i$.

\paragraph{Effect measures}

We will report odds ratios (ORs) and 95\% confidence intervals for $\exp(\beta_1)$ (and analogously $\exp(\gamma_1)$ for binary secondary outcomes). To support interpretation on an absolute scale, we will also report marginal predicted probabilities under $A=1$ and $A=0$ (and their difference). For multinomial outcomes, we will report contrasts relative to the reference category and marginal predicted category probabilities. For count outcomes, we will report rate ratios (or odds ratios if categorized) and marginal predicted means under $A=1$ and $A=0$ (and their difference).

\subsection{Connection to Hypotheses}

Each fitted model directly operationalizes the study hypotheses by testing whether exposure status is associated with healthcare access/utilization in the target population.

\begin{itemize}
    \item For the TBI analysis, we hypothesize worse healthcare access/utilization among adults with TBI. For outcomes coded as $1=\text{favorable}$ (e.g., insured; usual source of care), this corresponds to $\beta_1 < 0$ (OR $<1$).

    \item For the stroke analysis, we hypothesize greater healthcare contact and therefore more favorable access/utilization among adults with stroke. For outcomes coded as $1=\text{favorable}$, this corresponds to $\beta_1 > 0$ (OR $>1$).
\end{itemize}

Planned exploratory analyses of effect modification (e.g., by age, sex, race/ethnicity, education, and income-to-poverty ratio) will be conducted via exposure-by-subgroup interaction terms, with subgroup-specific estimates reported where supported by sample size.

\subsection{Multiple Testing}

Because we will evaluate multiple outcomes within each exposure analysis, we will control the familywise error rate using Bonferroni adjustments.

For a family of $m$ hypothesis tests, the Bonferroni-adjusted $p$-value will be computed as:
\[
p_{\text{Bonf}} = \min(m \times p_{\text{unadj}},\,1).
\]

We will present both unadjusted and Bonferroni-adjusted $p$-values for transparency. Primary inferential statements will be based on the adjusted $p$-values, while unadjusted $p$-values will be reported as descriptive evidence. Adjustments will be applied separately for each exposure (stroke; TBI) and \textbf{separately for the primary outcomes versus the secondary outcomes} (two families per exposure). Specifically, within each exposure analysis we will apply Bonferroni adjustment across the two primary outcomes as one family, and across the set of secondary outcomes as a separate family.

\subsection{Inferential Methods}

All tests will be two-sided. Unless otherwise stated, we will use a nominal familywise $\alpha = 0.05$ within each outcome family after Bonferroni adjustment.

\paragraph{Variance estimation and confidence intervals}

To obtain valid standard errors under the NHANES complex design, we will use survey design-based variance estimation (Taylor series linearization) that accounts for stratification and clustering, along with the analysis weights described above. We will compute 95\% confidence intervals on the log-odds scale and transform to the odds ratio scale for reporting. For multinomial and count models, confidence intervals will be computed on the model's natural link scale and then transformed to relative risk ratios/odds ratios or rate ratios as appropriate.

\paragraph{Software}

Survey-weighted regressions will be implemented using the \texttt{survey} framework in R (e.g., \texttt{svyglm} with \texttt{family=quasibinomial()} for binary outcomes), combined with the matched samples generated in Section~6. Multinomial and count models will be implemented using survey-compatible routines (or equivalent design-based approaches) with careful attention to sparse-cell stability.

\subsection{Challenges and Solutions}

Key anticipated analytic challenges and planned mitigations include:

\begin{itemize}
    \item \textbf{Residual imbalance:} If balance diagnostics indicate meaningful residual imbalance on prespecified covariates, we will fit covariate-adjusted outcome models as described in the primary analysis.

    \item \textbf{Multiplicity:} We will report both unadjusted and Bonferroni-adjusted $p$-values, with adjusted values guiding inferential claims, applying adjustments separately for primary and secondary outcome families within each exposure.

    \item \textbf{Sparse categories:} For multinomial outcomes with sparse cells, we will use prespecified collapses or dichotomizations to ensure stable estimation and interpretable contrasts, and will report the coding choices transparently.
\end{itemize}

% (Shell tables moved to Appendix~\ref{app:shell-tables}.)

\subsection{Interpretation and Generalizability}

\paragraph{Estimands}

Because matching is performed with an exposure-defined treated group, the primary estimand in the matched (overlap) sample is the average treatment effect on the treated (ATT). When combined analysis weights are applied (product of matching weights and rescaled NHANES interview weights), the estimand is the population average treatment effect on the treated (PATT) for the NHANES target population of exposed individuals meeting the study eligibility criteria, consistent with \citet{dugoff2014}. We will present both matched-sample and population-generalized estimates to transparently convey the role of overlap and generalization.

\paragraph{Substantive interpretation}

We will emphasize effect sizes and uncertainty (ORs, risk differences, and 95\% CIs) rather than binary ``statistical significance.'' Multiplicity-adjusted $p$-values will guide formal inferential statements for the prespecified outcome families.

\paragraph{Generalizability diagnostics}

To assess whether population generalization is empirically supported, we will summarize (i) the distribution of the combined weights (including extrema and effective sample size) and (ii) overlap diagnostics. If diagnostics indicate highly variable weights or limited overlap, we will interpret population-generalized estimates as applying primarily to the overlap subpopulation and will consider stabilized and/or truncated weights as sensitivity analyses (reported as such).

\section{Sensitivity Analysis}
%========================

\subsection{Formal Sensitivity Methods}

Because propensity score matching addresses only measured confounding, we will assess robustness of key inferences to potential \emph{unmeasured} confounding using Rosenbaum-style $\Gamma$ sensitivity analyses for matched observational studies. The sensitivity parameter $\Gamma \ge 1$ represents the degree to which, within a matched set, two individuals with identical observed covariates could differ in their odds of exposure due to an unobserved factor. When $\Gamma=1$, the study is consistent with no hidden bias; larger $\Gamma$ values represent increasing departures from ignorability.

We will conduct $\Gamma$ sensitivity analyses for the prespecified \textbf{primary outcomes} within each matched cohort (stroke; TBI). For binary outcomes, we will use matched-set (stratified) tests appropriate for variable-ratio matching (1:4) to evaluate how large $\Gamma$ would need to be for conclusions to change under increasing hidden bias. Sensitivity analyses will be reported for the matched design (ATT in the overlap sample); we will present them as complementary to the survey-weighted population-generalized analyses, because Rosenbaum-style methods are fundamentally design-based for matched sets.

\subsection{Interpretation of Formal Sensitivity Analyses}

For each primary outcome and exposure, we will report the largest value of $\Gamma$ for which the study's inferential conclusion is unchanged (e.g., the smallest $\Gamma$ at which the $p$-value would cross the prespecified threshold after multiplicity adjustment). Intuitively, larger critical $\Gamma$ values indicate greater robustness: it would take a stronger unmeasured bias in exposure assignment to explain away the observed association.

We will interpret these results descriptively as a robustness diagnostic rather than a definitive correction for bias. Sensitivity results will be discussed alongside covariate-balance diagnostics and the negative control balance check, with emphasis on how much unmeasured confounding would be required to materially alter conclusions for the primary outcomes.

\subsection{Stability Analyses}

We will assess the stability of results to reasonable analytic choices that do not change the scientific question. These may include:

\begin{itemize}
    \item \textbf{Alternative matching specifications (design-stage).} Re-estimating effects under alternative matching ratios (e.g., 1:1--1:6) and/or alternative calipers (e.g., 0.1, 0.2, 0.25 SD of the logit PS), restricted to specifications that maintain acceptable post-match covariate balance.

    \item \textbf{Outcome-model robustness.} Re-fitting primary outcome models with adjustment for any covariates that remain meaningfully imbalanced after matching (absolute SMD $>0.10$), as a robustness check to residual imbalance.

    \item \textbf{Coding robustness.} Evaluating whether conclusions are sensitive to minor, prespecified alternative codings of key derived covariates (e.g., alternative smoking-status categorization) and, where relevant, alternative codings of secondary outcomes (e.g., different cut points for ``time since last healthcare visit'').

    \item \textbf{Weight diagnostics.} Summarizing the distribution of combined analysis weights (NHANES rescaled interview weight multiplied by matching weight) and repeating analyses with stabilized and/or truncated weights as a sensitivity analysis if extreme weights are observed, reporting any such truncation choices transparently.
\end{itemize}

All sensitivity and stability analyses will be clearly labeled as such and will not replace the prespecified primary analysis.

%========================
\section{Amendments}
%========================
Any amendments to this protocol will be documented and reported in the final manuscript and/or supplementary materials. Amendments will include a description of the change, the rationale, the date implemented, and whether the change occurred before or after examining outcome data. Examples of amendable items include changes to inclusion/exclusion criteria, exposure or outcome definitions, propensity score specification, matching parameters, weighting/variance procedures, and sensitivity analyses.

%========================
\section{Funding}
%========================
This study has no external funding to report. There were no funders involved in the study design; data acquisition; analysis; interpretation; manuscript preparation; or the decision to submit results for publication.

%========================
\section{Future Steps}
%========================

After completion of the analyses described in this protocol, we will:
\begin{enumerate}
    \item Execute the prespecified matched, survey-weighted outcome analyses for stroke and TBI cohorts; generate all tables/figures (including descriptive summaries, balance diagnostics, and primary/secondary outcome results).
    \item Complete the prespecified sensitivity and stability analyses and compile diagnostics on overlap and weight distributions.
    \item Draft a Results section and Discussion consistent with STROBE reporting elements that depend on observed results (including presentation of main effect estimates and key findings), and prepare supplementary materials documenting diagnostics and any protocol amendments.
\end{enumerate}

\appendix
\section{Shell Tables and Reporting Templates}
\label{app:shell-tables}

The following tables provide reporting templates for descriptive summaries, covariate balance diagnostics, and outcome distributions. They are included to standardize reporting and do not define additional analyses beyond those specified in the main protocol.

%==================================================
% Shell tables
%==================================================

%------------------------------
% Table 1: Demographics (TBI)
%------------------------------
\begin{table}[htbp]
\centering
\caption{Demographic and clinical characteristics of adults with and without traumatic brain injury (NHANES 2011--2014 TBI subsample)}
\label{tab:demographics-tbi}
\begin{tabular}{lccc}
\hline
\textbf{Characteristic} & \textbf{Total} & \textbf{TBI} & \textbf{No TBI} \\
\hline
Age (years) & & & \\
Sex & & & \\
\quad Male & & & \\
\quad Female & & & \\
Race / Ethnicity & & & \\
Income-to-poverty ratio & & & \\
Health insurance coverage & & & \\
Education level & & & \\
Alcohol consumption & & & \\
Hypertension & & & \\
Diabetes & & & \\
Smoking & & & \\
Ever told you had a stroke? & & & \\
If yes: age at stroke (years) & & & \\
If yes: time since stroke (years) & & & \\
\hline
\end{tabular}
\end{table}

%------------------------------
% Table 2: Demographics (Stroke)
%------------------------------
\begin{table}[htbp]
\centering
\caption{Demographic and clinical characteristics of adults with and without stroke (NHANES 2011--2014 stroke subsample)}
\label{tab:demographics-stroke}
\begin{tabular}{lccc}
\hline
\textbf{Characteristic} & \textbf{Total} & \textbf{Stroke} & \textbf{No stroke} \\
\hline
Age (years) & & & \\
Sex & & & \\
\quad Male & & & \\
\quad Female & & & \\
Race / Ethnicity & & & \\
Income-to-poverty ratio & & & \\
Health insurance coverage & & & \\
Education level & & & \\
Alcohol consumption & & & \\
Hypertension & & & \\
Diabetes & & & \\
Smoking & & & \\
\hline
\end{tabular}
\end{table}

%------------------------------
% Table 3: Propensity covariate balance (BEFORE matching)
%   (TBI and Stroke in a single shell, as in outline)
%------------------------------
\begin{table}[htbp]
\centering
\caption{Covariate balance before propensity score matching for TBI and stroke}
\label{tab:ps-balance-before}
\begin{tabular}{lcccccc}
\hline
\textbf{Variable (\% or mean)} &
\multicolumn{3}{c}{\textbf{TBI vs No TBI}} &
\multicolumn{3}{c}{\textbf{Stroke vs No stroke}} \\
\cline{2-4} \cline{5-7}
 & \textbf{No TBI} & \textbf{TBI} & \textbf{SMD} &
   \textbf{No stroke} & \textbf{Stroke} & \textbf{SMD} \\
\hline
Age (years) & & & & & & \\
Age range (e.g.\ 40--49, 50--59, \dots) & & & & & & \\
Sex: Male & & & & & & \\
Sex: Female & & & & & & \\
Race / Ethnicity & & & & & & \\
Income-to-poverty ratio & & & & & & \\
Health insurance coverage & & & & & & \\
Education level & & & & & & \\
Alcohol consumption & & & & & & \\
Hypertension & & & & & & \\
Diabetes & & & & & & \\
Smoking & & & & & & \\
% Add any additional covariates used in the PS model
\hline
\end{tabular}
\end{table}

%------------------------------
% Table 4: Propensity covariate balance (AFTER matching)
%   (TBI and Stroke in a single shell, as in outline)
%------------------------------
\begin{table}[htbp]
\centering
\caption{Covariate balance after propensity score matching for TBI and stroke}
\label{tab:ps-balance-after}
\begin{tabular}{lcccccc}
\hline
\textbf{Variable (\% or mean)} &
\multicolumn{3}{c}{\textbf{TBI vs No TBI}} &
\multicolumn{3}{c}{\textbf{Stroke vs No stroke}} \\
\cline{2-4} \cline{5-7}
 & \textbf{No TBI} & \textbf{TBI} & \textbf{SMD} &
   \textbf{No stroke} & \textbf{Stroke} & \textbf{SMD} \\
\hline
Age (years) & & & & & & \\
Age range (e.g.\ 40--49, 50--59, \dots) & & & & & & \\
Sex: Male & & & & & & \\
Sex: Female & & & & & & \\
Race / Ethnicity & & & & & & \\
Income-to-poverty ratio & & & & & & \\
Health insurance coverage & & & & & & \\
Education level & & & & & & \\
Alcohol consumption & & & & & & \\
Hypertension & & & & & & \\
Diabetes & & & & & & \\
Smoking & & & & & & \\
% Add any additional covariates used in the PS model
\hline
\end{tabular}
\end{table}

%------------------------------
% Table 5: Access / utilization indicators by TBI status
%------------------------------
\begin{table}[htbp]
\centering
\caption{Healthcare access and utilization indicators by TBI status}
\label{tab:access-tbi}
\begin{tabular}{lccc}
\hline
\textbf{Measure of access / utilization} & \textbf{Total} & \textbf{TBI} & \textbf{No TBI} \\
\hline
Has routine place to go for healthcare & & & \\
Type of usual place (e.g.\ clinic, doctor's office, ER, other) & & & \\
Any health insurance coverage & & & \\
Type of insurance: government & & & \\
Type of insurance: private & & & \\
Type of place most often used for healthcare & & & \\
Time since last healthcare visit & & & \\
Seen a mental health professional in past year & & & \\
Number of overnight hospital stays (past year) & & & \\
Any time without health insurance in past year & & & \\
Plan covers prescription medicines & & & \\
Plan covers mental-health related services & & & \\
\hline
\end{tabular}
\end{table}

%------------------------------
% Table 6: Access / utilization indicators by stroke status
%------------------------------
\begin{table}[htbp]
\centering
\caption{Healthcare access and utilization indicators by stroke status}
\label{tab:access-stroke}
\begin{tabular}{lccc}
\hline
\textbf{Measure of access / utilization} & \textbf{Total} & \textbf{Stroke} & \textbf{No stroke} \\
\hline
Has routine place to go for healthcare & & & \\
Type of usual place (e.g.\ clinic, doctor's office, ER, other) & & & \\
Any health insurance coverage & & & \\
Type of insurance: government & & & \\
Type of insurance: private & & & \\
Type of place most often used for healthcare & & & \\
Time since last healthcare visit & & & \\
Seen a mental health professional in past year & & & \\
Number of overnight hospital stays (past year) & & & \\
Any time without health insurance in past year & & & \\
Plan covers prescription medicines & & & \\
Plan covers mental-health related services & & & \\
\hline
\end{tabular}
\end{table}

\bibliography{references}

\end{document}